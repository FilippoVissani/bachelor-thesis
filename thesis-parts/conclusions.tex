\chapter*{Conclusioni}
\addcontentsline{toc}{chapter}{Conclusioni}
\markboth{CONCLUSIONI}{CONCLUSIONI}
Traendo le conclusioni, in questa tesi sono stati presentati i risultati degli studi effettuati a proposito dello sviluppo di applicazioni Mixed Reality per HoloLens 2.

Il campo della realtà mista è molto ampio, può essere descritto secondo svariati punti di vista e avere infinite applicazioni.
La definizione stessa di realtà mista si perfeziona con l'introduzione di nuove tecnologie che realizzano quest'ultima.

Nei quattro capitoli di questa tesi viene più volte evidenziato il fatto che HoloLens 2 è un dispositivo unico nel suo genere, che introduce importanti novità tecniche.
Nessun altro dispositivo in commercio è in grado di rilevare l'ambiente e di proiettare ologrammi permettendo all'utente di interagirci come se fossero reali, nel modo in cui lo fa HoloLens 2.
La realtà mista, della quale fino a qualche anno fa si è solo parlato da un punto di vista teorico, ora può essere concretizzata grazie ad HoloLens 2.

Tuttavia, questa tecnologia ha ancora dei limiti, come gli strumenti di sviluppo, che sono in parte ancora in fase sperimentale.
Come detto nella Sezione \ref{sec:sezione45}, un ambiente di sviluppo creato su misura risolverebbe i problemi che si sono presentati durante le fasi di studio e di sviluppo.
Probabilmente nei prossimi anni verrà fornito un supporto migliore per lo sviluppo di queste applicazioni.

Bisogna anche considerare che al momento gli utenti che hanno la possibilità di utilizzare HoloLens non sono molti perché inizialmente è stato commercializzato solo negli Stati Uniti, per poi essere reso disponibile in altri paesi solo nel corso del 2020.
Probabilmente in futuro dispositivi come HoloLens verranno commercializzati su larga scala, raggiungendo un gran numero di utenti.

Alcune delle esperienze che la realtà mista permette in teoria non possono ancora essere realizzate concretamente, la telepresenza è un esempio.
Con il progresso tecnologico, le potenzialità dei dispositivi Mixed Reality verranno incrementate notevolmente, questo permetterà di realizzare tutte le esperienze che finora sono state presentate solo a livello teorico.
In futuro sarà possibile realizzare sistemi che ridurranno sempre di più il divario fra mondo reale e virtuale, fino a farlo scomparire del tutto.