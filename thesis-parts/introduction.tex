\chapter{Introduzione}
\markboth{INTRODUZIONE}{INTRODUZIONE}
Negli ultimi anni realtà virtuale e realtà aumentata hanno iniziato a diffondersi sempre più rapidamente.
Ogni anno vengono introdotte tecnologie che garantiscono esperienze sempre migliori.
Una di queste tecnologie è HoloLens 2, un dispositivo concepito da Microsoft per permettere esperienze di realtà mista.

In questa tesi si vogliono presentare i risultati degli studi effettuati sullo sviluppo di applicazioni Mixed Reality per HoloLens 2.

La realtà mista viene prima di tutto descritta da un punto di vista teorico, ma vengono anche definite architettura e struttura delle applicazioni relative.
L'obiettivo è quello di fornire una visione completa della Mixed Reality, non solo da un punto di vista teorico, ma anche tecnico e pratico.

Nel primo capitolo, dopo l'introduzione di alcuni cenni storici, vengono presentati i principali aspetti della realtà mista, che viene anche comparata alla realtà virtuale e alla realtà aumentata.
Successivamente vengono descritte la caratteristiche che un'applicazione Mixed Reality ha.
Infine, vengono introdotti i dispositivi attualmente in commercio che supportano questa tecnologia e i framework di sviluppo relativi.

Il secondo capitolo introduce HoloLens 2 da un punto di vista tecnico, insieme all'architettura e allo sviluppo di applicazioni Mixed Reality per questo dispositivo.
In questo capitolo vengono evidenziate le potenzialità del dispositivo e viene anche fornito un primo esempio di applicazione.

Nel terzo capitolo vengono descritti alcuni ambiti in cui l'applicazione di HoloLens 2 può migliorare determinati aspetti, come la produttività, l'efficienza e la sicurezza.

Per mostrare in pratica alcune delle potenzialità della Mixed Reality descritte nei capitoli precedenti, nel quarto capitolo viene presentato il caso di studio, relativo all'ambito sanitario. Qui viene fornita una descrizione dettagliata del caso, corredata dall'analisi dei requisiti e da una proposta di progetto.
Infine, viene definito lo sviluppo del prototipo relativo al progetto, in modo da rendere il lettore cosciente di cosa può fare HoloLens 2 concretamente. 